\documentclass[11pt,letterpaper]{article}
\usepackage[utf8]{inputenc}
\usepackage[T1]{fontenc}
\usepackage{lmodern}
\usepackage[margin=0.5in]{geometry}
\usepackage{titlesec}
\usepackage{enumitem}
\usepackage{hyperref}
\usepackage{xcolor}
\usepackage{ragged2e}

% Custom section command
\titleformat{\section}
  {\large\bfseries\uppercase}
  {}{0em}{}[\titlerule]
\titlespacing*{\section}{0pt}{*1.5}{*0.5}

% Remove page numbers
\pagenumbering{gobble}

% Custom commands
\newcommand{\name}[1]{
  \begin{center}
    \Huge\textbf{#1}
  \end{center}
  \vspace{-0.5em}
  \hrule height 0.4pt width \textwidth
  \vspace{0.5em}
}
\newcommand{\contact}[5]{
  \begin{center}
    #1 | #2 | #3 \\
    #4 | #5
  \end{center}
  \vspace{-0.5em}
  \hrule height 0.4pt width \textwidth
  \vspace{0.5em}
}

\newcommand{\spacedSection}[1]{
  \vspace{0.1em}
  \section{#1}
  \vspace{1em}
}

\begin{document}

\name{Emanuel Jureczko}
\vspace{-0.5em}
\contact{+48 518 734 135}{Informatikstudent}{Gliwice, Polen}
{\href{mailto:emajure27@gmail.com}{emajure27@gmail.com}}{\href{https://github.com/jemek27}{github.com/jemek27}}

\vspace{-0.5em}
\begin{justify}
\textit{Drittsemester-Student mit solider theoretischer Grundlage und Erfahrung in der Hoch- und Niedrigsprachenprogrammierung.
In meiner Freizeit arbeite ich gerne an Nebenprojekten, um neue Technologien zu erlernen und meine Fähigkeiten zu entwickeln.
Ich suche nun den Übergang von persönlichen und Studienprojekten zu kommerziellen Projekten, bei denen ich mein Wissen
in einem professionellen Umfeld anwenden, zu realen Anwendungen beitragen und mit einem Team zusammenarbeiten kann, um wirkungsvolle Lösungen zu liefern.}
\end{justify}

\spacedSection{Fachgebiete}

\begin{minipage}[t]{0.33\textwidth}
    \begin{flushleft}
        \begin{itemize}[leftmargin=0.5cm]
            \setlength\itemsep{0.4em}
            \item Programmiersprachen – \textbf{Java, C, C++, C\#, Asm, SQL, Python, JavaScript}
            \item Versionskontrolle - \textbf{GitHub}
        \end{itemize}
    \end{flushleft}
\end{minipage}
\begin{minipage}[t]{0.33\textwidth}
    \begin{flushleft}
        \begin{itemize}[leftmargin=0.5cm]
        \setlength\itemsep{0.4em}
            \item Backend – \textbf{Spring Boot}
            \item Frontend – \textbf{HTML, CSS, JS}
        \end{itemize}
    \end{flushleft}
\end{minipage}
\begin{minipage}[t]{0.33\textwidth}
    \begin{flushleft}
        \begin{itemize}[leftmargin=0.5cm]
        \setlength\itemsep{0.4em}
            \item Datenbanken – \textbf{PostgreSQL}
            \item Container – \textbf{Docker}
        \end{itemize}
    \end{flushleft}
\end{minipage}

\vspace{1em}
\spacedSection{Erfahrung}

\textbf{Praktikum - Fullstack-Entwickler} \hfill September 2024 \ [0.4em]
RED Electronics \hfill \textit{Gliwice, Polen}
\begin{itemize}[leftmargin=0.5cm]
    \itemsep-3pt {}
    \item Entwicklung einer Webanwendung zur Datenvisualisierung und Geräteverwaltung.
    Projekt auf GitHub im Webapp-Repository.
    \begin{itemize}[leftmargin=0.5cm]
        \itemsep-3pt {}
        \item Frontend: HTML, CSS, JavaScript
        \item Backend: Node.js, \textbf{Python} (LoRa Daten Ein-/Ausgabe)
        \item Datenbank: \textbf{PostgreSQL}
    \end{itemize}
\end{itemize}

\spacedSection{Ausbildung}

\begin{itemize}[leftmargin=0.5cm]
\itemsep-3pt {}
    \item Informatik, Schlesische Technische Universität Gliwice \hfill {2022 - Gegenwart}
    \item Grafik und Digitaldrucktechnik, CKZiU Strzelce Opolskie \hfill {2018 - 2022}
\end{itemize}

\spacedSection{Sprachen}

\begin{itemize}[leftmargin=0.5cm]
    \itemsep-3pt {}
    \item Deutsch – B2/C1 (Fortgeschritten) – TestDaF-Zertifikat: 4/4/3/4
    \item Englisch – B2 (Fortgeschritten)
    \item Polnisch – Muttersprache
\end{itemize}

\spacedSection{Projekte}

\begin{itemize}[leftmargin=0.5cm]
\setlength\itemsep{0.4em}
    \item \textbf{Ecommerce App} \\
    Ein in Entwicklung befindliches Projekt, um moderne Technologien zu erkunden und anzuwenden.
    Das Projekt soll mein Wissen über Backend-Entwicklung, Sicherheit und Deployment-Praktiken vertiefen.\ [.4em]
    \textit{Technologien:} \textbf{Java, Spring Boot, Spring Data JPA, PostgreSQL, Docker, RESTful API, Spring Security + JWT}

    \item \textbf{Gaussian Blur ASM vs C++} \\
    Entwickelt im 5. Semester als Teil eines Assemblersprachenkurses.
    Es konzentriert sich auf den Vergleich der Ausführungszeiten zwischen benutzerdefinierten C++- und Assembly-Bibliotheken.
    Die implementierten Bibliotheken wenden einen Gaußschen Weichzeichner auf ein Bild an.\\ [.3em]
    \textit{Technologien:} \textbf{Asm, C++, C\#}

    \item \textbf{Sudoku Solver} \\
    Ein Sudoku-Löser, der menschliche Lösungsstrategien nachahmt, anstatt brute-force zu verwenden.\\ [.3em]
    \textit{Technologien:} \textbf{C++}

    \item \textbf{Webapp} \\
    Teil eines Praktikumsprojekts zur LoRa-basierten Kommunikation, erstellt bei RED Electronics.
    Diese Anwendung verarbeitet empfangene Daten, visualisiert sie und ermöglicht das Senden von Steuersignalen an die Sensoreinheit.\\ [.3em]
    \textit{Technologien:} \textbf{HTML, CSS, JavaScript, Node.js, Python, PostgreSQL}

    \item \textbf{Pac-Man} \\
    Programmierprojekt im 4. Semester. Eigene Implementierung von Pac-Man mit SFML 2.5.1 cpp \\ [.3em]
    \textit{Technologien:} \textbf{C++ SFML}

    \item \textbf{Console Snake} \\
    Ein einfaches Snake-Spiel, das in der Windows-Konsole läuft.\\ [.3em]
    \textit{Technologien:} \textbf{C}

\end{itemize}

\end{document}
